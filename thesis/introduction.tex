\chapter{Introduction}

% rapidly increasing life span
% more AD cases
% therefore research into brain aging process



The past decades have seen medical research progress at a rapid rate, resulting in a dramatic increase of the average human life expectancy \cite{owidlifeexpectancy}. As a consequence, the number of cases of aging-related neurodegenerative diseases such as Alzheimer's Disease has increased significantly and is expected to keep rising, reaching over 100 million Alzheimer's cases by 2050 \cite{brookmeyer2007forecasting}. For this reason, advancing the understanding of the brain aging process as well as early prediction and treatment methods of degenerative diseases have attracted considerable research efforts. In our work, we propose a generative model to simulate the aging process on T1-weighted MRI brain scans.

Generative models, most prominently Variational Autoencoders (VAEs) \cite{kingma2013auto} and Generative Adversarial Networks (GANs) \cite{goodfellow2014generative}, have been successfully applied to model gradual changes observed in image data for a wide range of settings such as face aging \cite{palsson2018generative}, image registration \cite{balakrishnan2019voxelmorph} and style transfer \cite{zhu2017unpaired}. In this thesis, we propose a diffeomorphic Generative Adversarial Network architecture to model the brain aging process. Specifically, we consider a model which given a T1-weighted MRI image $x$ taken at time $t_0$ predicts an image $y$ at some time $t_1$ in the future. The ability to predict follow-up images in this manner could pave the way for a number of applications. For instance, existing diagnostic tools such as diagnosis classifiers operating on MRI data can be applied directly to the generated image, therefore leveraging decades of research in the field. Furthermore, the generative aging model has applications in the context of visual feature attribution, for instance highlighting the areas of the brain most strongly correlated with a degenerative condition.

In the field of medical imaging and computational anatomy, diffeomorphic deformations are a popular model choice \cite{beg2005computing} \cite{ashburner2007fast}. Unlike entirely convolutional models, diffeomorphisms are constrained to invertible and therefore topology preserving transformations and are thus generally better suited to model the gradual changes observed in sensitive tissues such as the brain. Following \cite{dalca2018unsupervised}, we train a model to generate a diffeomorphic deformation field which we obtain by numerically integrating a stationary velocity field using the \textit{scaling and squaring} method \cite{arsigny2006log}. In its original formulation, the scaling and squaring method is used to integrate a vector field to one specific time step. In order to maximize the available training data as well as being able to generalize over larger time ranges, we therefore propose a modification to the method to approximate deformations for arbitrary time steps. In turn, this allows our brain aging model to generate, and be trained on, image pairs with arbitrary time differences. 

Finally, validating generative model outputs beyond subjective visual inspection is a notoriously difficult task. While user studies can be employed in some domains such as face aging \cite{palsson2018generative}, this is not a viable option for the task of brain aging. Instead, we propose to use a pre-trained age regressor applied to our generator's outputs as a more meaningful and comparable metric. We show that while an age regressor's absolute loss may be comparatively high, the relative loss between two images taken from the same subject can be significantly lower due to an error cancelling effect.

Our main contributions are:

\begin{itemize}
	\item we propose a modification to the scaling and squaring method for arbitrary timesteps
	\item we model the brain aging process using diffeomorphic deformations%, adapting the architecture proposed by \cite{dalca2018unsupervised}
	\item we suggest using a pre-trained age regressor to validate our generative model's ability to predict follow-up images
	%\item we validate our model's ability to predict follow-up images using a pre-trained age regressor
	\item we demonstrate the effectiveness of our model in the context of MCI to AD conversion prediction
\end{itemize}

\section{Related Work}
Most similar to our work in terms of the problem setting, \cite{wegmayr2019generative} use a WGAN architecture to model the brain's aging process. They propose a UNet-derived \cite{ronneberger2015unet} model architecture based on \cite{baumgartner2018visual}, which is trained and applied iteratively to obtain predictions for different time steps. In addition to modeling the differences in the aging process between healthy subjects and subjects affected by Alzheimer's Disease, they report cautiously positive results for the task of conversion prediction.
\cite{baumgartner2018visual} use a WGAN on 3D MRI brain data to perform visual feature attribution and apply it to generate image-specific effect maps of Alzheimer's Disease.
In contrast to our work, the models used in \cite{wegmayr2019generative} and \cite{baumgartner2018visual} are purely convolutional and do not use deformations.

Pursuing a goal similar to ours, \cite{pathan2018predictive} use a combination of recursive and convolutional neural networks to predict a sequence of deformations based on, and then applied to, a baseline image to obtain follow-up predictions at different time steps. The model is trained on the first image of a subject as well as the sequence of diffeomorphic vector momenta for each additional image which are generated using the LDDMM framework \cite{beg2005computing}.

From a model perspective, the work most similar to ours is \cite{balakrishnan2019voxelmorph} \cite{dalca2018unsupervised}, which forms the basis of our work both in terms of the model design as well as its implementation. While the architecture was initially introduced on the task of unsupervised image registration, it has since been adapted to the problem of unsupervised segmentation \cite{dalca2019unsupervised}.
In the domain of face aging, \cite{palsson2018generative} suggest a conditional GAN \cite{mirza2014conditional} architecture similar to ours which uses an age regressor as part of its loss function.

As previously mentioned, early prediction of Alzheimer's Disease onset has been the target of considerable research efforts. \cite{thung2016identification} propose an SVM classifier distinguishing subjects with stable and progressive MCI and report an accuracy of 78.2\%. In contrast to our work, the features are extracted from longitudinal MRI brain data collected over a time period of up to 18 months.
Reporting an accuracy of 92\%, \cite{sun2017detection} also use an SVM classifier on features extracted from a stationary velocity field, which is calculated using \cite{vercauteren2009diffeomorphic} on longitudinal data of up to 36 months.


