\chapter{Introduction}

% rapidly increasing life span
% more AD cases
% therefore research into brain aging process



The past fifty years have seen medical research advancing at a rapid rate, resulting in a dramatic increase of the average human life expectancy. As a consequence, the number of cases of aging-related neurodegenerative diseases such as Alzheimer's Disease has increased significantly and is expected to keep rising, reaching over 100 million Alzheimer's cases by 2050. As such, advancing the understanding of the brain's aging process as well as early prediction and treatment methods of degenerative diseases have attracted considerable research efforts (TODO list some). In our work, we propose a generative model to simulate the aging process on T1-weighted MRI brain scans based on diffeomorphic deformations.

 

Generative models, most prominently Variational Autoencoders (VAEs) and Generative Adversarial Networks (GANs), have been successfully used to model gradual changes in image data for wide range of domains such as face aging, image registration and .... The goal of this thesis is to apply some of these methods to the problem of brain aging. Specifically, we consider a T1-weighted MRI scan x taken at time t0 and aim to predict from this inital image a new scan y at some specified time t1 in the future. 





main contributions

model the aging process of the human brains using diffeomorphic deformations
adapting the scaling and squaring method for arbitrary timesteps

Our main contributions are:

\begin{itemize}
	\item we model the brain's aging process as diffeomorphic deformations, adapting the architecture proposed by \cite{voxelmorph}
	\item we adapt the Scaling and Squaring method for arbitrary timesteps

	\item we use the generative model to predict the converion probability of patients with MCI

	\item we validate that the model does indeed result in aged images using an age regressor


\end{itemize}
